\usepackage{epigraph}
\usepackage{fancyhdr}
\usepackage{fontspec}
\usepackage{emptypage}
\setlength{\parindent}{0pt}
\newcommand*{\blankpage}{
{\newpage \vspace*{5cm}\thispagestyle{empty}\centering \bfseries  \textit{This page is intentionally left blank.} \par}
\vspace{\fill}}
\let\uwu=\cleardoublepage

%titlepage
\renewcommand\epigraphflush{flushright}
\renewcommand\epigraphsize{\normalsize}
\setlength\epigraphwidth{0.7\textwidth}

\DeclareFixedFont{\titlefont}{T1}{ppl}{bx}{n}{0.75in}

\makeatletter                       
\def\printauthor{%                  
    {\large \@author}}              
\makeatother
\author{Bustos Jordi\\
    \texttt{jordibustos01@gmail.com}
    }

\newcommand\titlepagedecoration{%
\begin{tikzpicture}[remember picture,overlay,shorten >= -10pt]

\coordinate (aux1) at ([yshift=-15pt]current page.north east);
\coordinate (aux2) at ([yshift=-410pt]current page.north east);
\coordinate (aux3) at ([xshift=-4.5cm]current page.north east);
\coordinate (aux4) at ([yshift=-150pt]current page.north east);

\begin{scope}[black!40,line width=12pt,rounded corners=12pt]
\draw
  (aux1) -- coordinate (a)
  ++(225:5) --
  ++(-45:5.1) coordinate (b);
\draw[shorten <= -10pt]
  (aux3) --
  (a) --
  (aux1);
\draw[opacity=0.6,black,shorten <= -10pt]
  (b) --
  ++(225:2.2) --
  ++(-45:2.2);
\end{scope}
\draw[black,line width=8pt,rounded corners=8pt,shorten <= -10pt]
  (aux4) --
  ++(225:0.8) --
  ++(-45:0.8);
\begin{scope}[black!70,line width=6pt,rounded corners=8pt]
\draw[shorten <= -10pt]
  (aux2) --
  ++(225:3) coordinate[pos=0.45] (c) --
  ++(-45:3.1);
\draw
  (aux2) --
  (c) --
  ++(135:2.5) --
  ++(45:2.5) --
  ++(-45:2.5) coordinate[pos=0.3] (d);   
\draw 
  (d) -- +(45:1);
\end{scope}
\end{tikzpicture}%
}

\usepackage{framed}
\usepackage{titletoc}

\patchcmd{\tableofcontents}{\contentsname}{\sffamily\contentsname}{}{}

\renewenvironment{leftbar}
  {\def\FrameCommand{\hspace{6em}%
    {\color{black}\vrule width 2pt depth 6pt}\hspace{1em}}%
    \MakeFramed{\parshape 1 0cm \dimexpr\textwidth-6em\relax\FrameRestore}\vskip2pt%
  }
 {\endMakeFramed}



\titlecontents{chapter}
  [0em]{\Large\bfseries\vspace*{2\baselineskip}}
  {\hspace{-1.2cm}\parbox{6.5em}{%
    \hfill\Huge\sffamily\bfseries\color{black}\thecontentspage}%
   \vspace*{-1.9\baselineskip}\leftbar\sffamily\bfseries}
  {}{\endleftbar}
\titlecontents{section}
  [11.4em]
  {\sffamily\contentslabel{3em}}{}{} 
  {\phantom{    }\dotfill \nobreak\itshape\color{black}\contentspage}
\titlecontents{subsection}
  [13em]
  {\sffamily\contentslabel{3em}}{}{}  
  {\phantom{    }\dotfill \nobreak\itshape\color{black}\contentspage}


\newcommand{\mychapter}[2]{
    \setcounter{chapter}{#1}
    \setcounter{section}{0}
    \chapter*{#2}
    \addcontentsline{toc}{chapter}{#2}
}


\usepackage{titlesec}

\titleformat{\chapter}
{\vspace{-1.2cm}\normalfont\fontsize{35}{25}\bfseries}{}{0pt}{}

\titleformat{\section}
{\normalfont\Large\bfseries}{~\thesection}{1em}{}

\usepackage{fancyhdr}

% Configuración de la página y el encabezado
\pagestyle{fancy}
\fancyhf{}  % Borra todos los encabezados y pies de página actuales

% Encabezado en páginas pares (izquierda)
\fancyhead[LE]{\thepage \hspace{0.2cm}  $\bullet$ \hspace{0.2cm} \nouppercase{\rightmark}}
% Encabezado en páginas impares (derecha)
\fancyhead[RO]{\nouppercase{\rightmark} \hspace{0.2cm} $\bullet$ \hspace{0.2cm} \thepage}

\renewcommand{\headrulewidth}{0pt}  % Línea horizontal en el encabezado

% Modificación para mostrar solo el nombre de la sección sin el número
\renewcommand{\sectionmark}[1]{\markright{\MakeUppercase{#1}}}


\setlength{\headheight}{14.5pt}
\addtolength{\topmargin}{-2.5pt}

