%!TEX root = main.tex
\usepackage[T1]{fontenc}
\usepackage[spanish]{babel}
\usepackage{graphicx}
\usepackage{float}
\usepackage{booktabs}
\usepackage{enumitem}
\usepackage{subcaption}
\usepackage{multicol}
\usepackage[usenames,dvipsnames]{xcolor}
\usepackage{amsmath, amsfonts, mathtools, amsthm, amssymb}
\usepackage{mathrsfs}
\usepackage{euler}
\usepackage{cancel}
\usepackage{enumitem}
\usepackage[breaklinks = true]{hyperref}
\usepackage{lipsum}

\newcommand\N{\ensuremath{\mathbb{N}}}
\newcommand\R{\ensuremath{\mathbb{R}}}
\newcommand\Z{\ensuremath{\mathbb{Z}}}
\renewcommand\O{\ensuremath{\varnothing}}
\newcommand\Q{\ensuremath{\mathbb{Q}}}
\newcommand\C{\ensuremath{\mathbb{C}}}
\newcommand\e{\ensuremath{\varepsilon}}
\newcommand{\mf}[1]{\mathfrak{#1}}
\newcommand{\mc}[1]{\mathcal{#1}}

\newcommand\restr[2]{{% we make the whole thing an ordinary symbol
\left.\kern-\nulldelimiterspace % automatically resize the bar with \right
#1 % the function
\littletaller % pretend it's a little taller at normal size
\right|_{#2} % this is the delimiter
}}

\newcommand{\littletaller}{\mathchoice{\vphantom{\big|}}{}{}{}}

\newcommand{\norm}[1]{\left\lVert#1\right\rVert}
% \norm{ \biggl(\sum_{n=1}^N \mathbf{P}_{n}\biggr) }

\def\upint{\mathchoice%
  {\mkern13mu\overline{\vphantom{\intop}\mkern7mu}\mkern-20mu}%
  {\mkern7mu\overline{\vphantom{\intop}\mkern7mu}\mkern-14mu}%
  {\mkern7mu\overline{\vphantom{\intop}\mkern7mu}\mkern-14mu}%
  {\mkern7mu\overline{\vphantom{\intop}\mkern7mu}\mkern-14mu}%
  \int}
\def\lowint{\mkern3mu\underline{\vphantom{\intop}\mkern7mu}\mkern-10mu\int}

\usepackage{tikz}
\usepackage{pgfplots}

\pgfplotsset{compat=1.18}

\usepackage{geometry}
\geometry{
  a4paper,
  total={170mm,245mm},
  left=20mm,
  top=20mm,
}

\input{Configs/Boxes}
\input{Configs/Formats}
