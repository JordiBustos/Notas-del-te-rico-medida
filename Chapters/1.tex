\section{Integral de Riemann}

Sea $f: [a, b] \subseteq \R \to \R$ una función.
Una partición $P$ de $[a, b]$ es un conjunto finito $\{ x_0, x_1, \cdots, x_n \}$, con
$a = x_0 < x_1 < \cdots < x_n = b$. A $P$ le asignamos una norma $\| P \| = \max \{ l(J_k) \}$. $J_k = [x_{k-1}, x_k]$ y a cada $P$ le
podemos asignar una etiqueta, que es un vector $\xi = (\xi_1, \cdots, \xi_n)$ tal que $\xi_k \in J_k$. Una partición etiquetada es un par
$(P, \xi)$; y le podemos asignar su suma de Riemann: $S(P, \xi) = \sum_{k=1}^{n} f(\xi_k) l(J_k)$.

\begin{definition}
    Una función $f: [a, b] \to \R$ es integrable Riemann si $\exists I \in \R : \forall \e > 0, \exists \delta > 0 : | S(P, \xi) - I | < \e$ si $(P, \xi)$ es tal que $\| P \| \leq \delta$
\end{definition}

Ejercicio: Probar que si $f$ es integrable Riemann entonces es acotada.

Si $f$ es acotada, dada una partición $P$ del dominio de $f$, para cada $i \in 1, \cdots, n$. Sean $M_i = sup \{ f(x) : x \in J_i \}$ y
$m_i = inf\{ f(x) : x \in J_i \}$. Luego definimos la suma superior y la suma inferior asociada a $P$ como $S(f, P) = \sum_{k=1}^{n} M_k l(J_k)$ y
$s(f, P) = \sum_{k=1}^{n} m_k l(J_k)$. Entonces podemos definir suma superior e inferior de Riemann como
$\lowint_a^b f(x) \, \mathrm{d}x = sup \{ S(f, P) : P \text{ partición de } [a, b] \}$ y
$\upint_a^b f(x) \, \mathrm{d}x = inf \{ s(f, P) : P \text{ partición de } [a, b] \}$.

\begin{prop}
    Dada una función $f: [a, b] \to \R$, $f$ es integrable Riemann $\iff$ es acotada y la suma superior es igual a la inferior.
\end{prop}

\begin{note}
    $f$ es integrable Riemann si: \begin{enumerate}
        \item $f$ es continua.
        \item $f$ es continua salvo finitos puntos en los que existen los límites laterales.
        \item $f$ es monótona y acotada (en este caso pueden existir numerables discontinuidades).
    \end{enumerate}
\end{note}

\subsection{Desventajas de la integral de Riemann}
\begin{itemize}
    \item Exige que la función oscile poco en intervalos pequeños.
    \item Hay funciones simples que no son integrables Riemann.
    \item No se comporta bien con respecto a la convergencia puntual.
\end{itemize}

\begin{eg}
    Sea $f: [0, 1] \to \R : f(x) = \begin{cases}
            1 & x \in \Q              \\
            0 & x \in \R \setminus \Q
        \end{cases} f$ no es integrable Riemann.
    \begin{proof}
        Llamemos $A = [0, 1] \bigcap \Q$. $A$ es numerable entonces $\exists \sigma: \N \to A$ biyección. Para cada $n \in \N$, sea
        $A_n = \{ \sigma(1), \cdots, \sigma(n) \}$, $A_n \subset A_{n+1}$ y $\bigcup_{n=1}^{\infty} A_n = A$.
        Ahora para cada $n \geq 1$ consideramos:
        $f_n: [0, 1] \to \R$ dada por
        \begin{equation}
            f_n(x) = \begin{cases}
                1 & x \in A_n                  \\
                0 & x \in [0, 1] \setminus A_n
            \end{cases}
        \end{equation}
        $f_n$ es integrable Riemann (queda como ejercicio demostrarlo) ya que es continua salvo en los puntos de $A_n$ y
        los límites laterales son siempre cero.
        Veamos ahora que $f_n \to f$. Sea $x \in [0, 1]$ \begin{enumerate}
            \item Si $x \in A \to x \in A_{n_0}, n_0 \in \N \to (\forall n > n_0) x \in A_n \to (\forall n > n_0) f_n(x) = 1 \to f_n(x) \to f(x) = 1$.
            \item Si $x \notin A \to (\forall n \in \N) x \notin A_n \to (\forall n \in \N) f_n(x) = 0 \to f_n(x) \to f(x) = 0$.
        \end{enumerate} $\therefore f_n \to f$.
        Si conocieramos $l(A)$ y $l([0, 1] \setminus A)$ podríamos definir $\int f = 1 \times l(A) + 0 \times l([0, 1] \setminus A)$.
    \end{proof}
\end{eg}

\section{Espacios Medibles}

Dado $X$ un conjunto arbitrario no vacío. Sea $\mathcal{P}(X)$ el conjunto de partes de $X$.

\begin{definition}[$\sigma$-álgebra]
    Una familia $\mathfrak{X}$ es una $\sigma$-álgebra si verifica:
    \begin{enumerate}
        \item $\emptyset, X \in \mathfrak{X}$.
        \item Si $A \in X \to A^c \in \mathfrak{X}$.
        \item Sea $(A_n)_{n \geq 1}$ es una sucesión en $\mathfrak{X} \to \bigcup_{n=1}^{\infty} A_n \in \mathfrak{X}$.
    \end{enumerate}
\end{definition}

Si $\mathfrak{X}$ es una $\sigma$-álgebra de subconjuntos de $\mathfrak{X}$ el par $(X, \mathfrak{X})$ es un espacio medible. A cada $A \in \mathfrak{X}$
lo llamaremos conjunto $\mathfrak{X}$-medible.

\begin{note}
    Si $\mathfrak{X}$ es una $\sigma$-álgebra de $X$ y $A_1, \cdots A_n \in \mathfrak{X}$ entonces $\bigcup_{k=1}^{n} A_k \in \mathfrak{X}$.
    Idea de la demostración: Sea $(B_m)_{m \geq 1}$ la sucesión en $\mathfrak{X}$ definida por \begin{equation}
        B_m = \begin{cases}
            A_m       & 1 \leq m \leq n \\
            \emptyset & m > n
        \end{cases}
    \end{equation}
\end{note}

\begin{note}
    Si $(A_n)_{n \geq 1}$ es una sucesión de una $\sigma$-álgebra $\mathfrak{X}$ entonces $\bigcap_{n=1}^{\infty} A_n \in \mathfrak{X}$.
    \begin{proof}
        $\bigcup_{n \geq 1} A_n^c \in \mathfrak{X} \to (\bigcap_{n \geq 1} A_n^c)^c \in \mathfrak{X} \to \bigcup_{n \geq 1} A_n \in \mathfrak{X}$.
    \end{proof}
\end{note}

\begin{eg}[$\sigma$-álgebras]
    Dado $X$ cualquiera no vacío.
    \begin{enumerate}
        \item $\mathfrak{X} = \{ \emptyset, X \}$ es una $\sigma$-álgebra.
        \item $\mathfrak{X} = \mathcal{P}(X)$ es una $\sigma$-álgebra.
        \item Sea $A \neq \emptyset \subset X$. Luego $\mathfrak{X} = \{ \emptyset, A, A^c, X \}$ es una $\sigma$-álgebra.
        \item Supongamos que $X$ no es numerable y sea
              \begin{equation}
                  \mathfrak{X} = \{ A \subseteq X : A \text{ es numerable ó } A^c \text{ es numerable} \}
              \end{equation}
              es una $\sigma$-álgebra. Demostración ejercicio y además $\mathfrak{X} \neq \mathcal{P}(X)$.
    \end{enumerate}
\end{eg}

\begin{lemma}
    Dado un conjunto $X$, sean $\mathfrak{X}_1, \mathfrak{X}_2$ dos $\sigma$-álgebras de $X$. Entonces $\mathfrak{X}_1 \cap \mathfrak{X}_2$ es una $\sigma$-álgebra de $X$.
    Más aún si $(\mathfrak{X}_i)_{i \in I}$ es una familia de $\sigma$-álgebras de $X$ entonces $\bigcap_{i \in I} \mathfrak{X}_i$ es una $\sigma$-álgebra de $X$.
    Demostración, ejercicio.
\end{lemma}

\begin{prop}
    Dado un conjunto $X$, sea $A \neq \emptyset \subseteq \mathcal{P}(X) \to \exists \sigma$-álgebra $\sigma(A)$ que verifica: \begin{enumerate}
        \item $A \subseteq \sigma(A)$.
        \item $\mathfrak{X}$ es $\sigma$-álgebra de $X$ tal que $A \subseteq X \to \sigma(A) \subseteq \mathfrak{X}$.
        \item $\sigma(A)$ es la única que verifica ambas propiedades en simultáneo.
    \end{enumerate}
    La llamaremos $\sigma$-álgebra generada por $A$.

    \begin{proof}
        Sea $\Delta = \{ \mathscr{C} \subseteq \mathcal{P}(X) : \mathscr{C} \text{ es } \sigma \text{-álgebra de } X \text{ y } A \subseteq \mathscr{C} \} \neq \emptyset$ pues $\mathcal{P}(X) \in \Delta$.
        Llamemos $\mathfrak{X} = \bigcap_{\mathscr{C} \in \Delta} \mathscr{C} = \{ B \in \mathcal{P}(X) : B \in \mathscr{C} (\forall \mathscr{C} \in \Delta) \}$.
        Veamos que $\mathfrak{X}$ es una $\sigma$-álgebra de $X$.
        \begin{enumerate}
            \item $\emptyset, X \in \mathscr{C} (\forall \mathscr{C} \in \Delta) \to \emptyset, X \in \mathfrak{X}$.
            \item Sea $A \in \mathfrak{X} \to (\forall \mathscr{C} \in \Delta) A \in \mathscr{C} \to A^c \in \mathscr{C} (\forall \mathscr{C} \in \Delta) \to A^c \in \mathfrak{X}$.
            \item Sea $(A_n)_{n \geq 1}$ una sucesión en $\mathfrak{X}$ el argumento es análogo a los dos anteriores.
        \end{enumerate}
        $\therefore$ $\mathfrak{X}$ es una $\sigma$-álgebra que verifica ambas condiciones.
        Supongamos que existe otra $\overline{\mathfrak{X}}$ $\sigma$-álgebra que verifica las dos condiciones por la propiedad uno y dos podemos deducir que $\mathfrak{X} \subseteq \overline{\mathfrak{X}}$ y $\overline{\mathfrak{X}} \subseteq \mathfrak{X}$.
    \end{proof}
\end{prop}

\begin{eg}
    Consideremos $X = \R$ y sea $A = \{ (a, b) : a, b \in \R, a \leq b \}$. La $\sigma$-álgebra generada por $A$ es la $\sigma$-álgebra de Borel $\mathcal{B}$.
    A los conjuntos de $\mathcal{B}$ los llamaremos conjuntos Borelianos. Veamos que si $\overline{A} = \{ (a, +\infty) : a \in \R \} \to \sigma(\overline{A}) = \mathcal{B}$.
    \begin{proof}
        \begin{itemize}
            \item Dado $a \in \R$, $(a, +\infty) = \bigcup_{n \geq 1} (a, a+n) \in \mathcal{B} \to \overline{A} \subseteq \mathcal{B}$.
                  Luego $\sigma(\overline{A}) \subseteq \mathcal{B}$. Por ser $\sigma(\overline{A})$ la mínima $\sigma$-álgebra que contiene a $\overline{A}$.
            \item Dado $a, b \in \R$, $a < b$. Sabemos que $(a, b] = (a, +\infty) \bigcap (b, +\infty)^c \in \sigma(\overline{A})$.
                  Luego $(a, b) = \bigcup_{n \geq 1} (a, b - \frac{1}{n}] \in \sigma(\overline{A})$. Por lo que $A \subset \sigma(\overline{A})$.
                  $\mathcal{B} = \sigma(A) \subset \sigma(\overline{A})$. Por ser $\sigma(A)$ la mínima $\sigma$-álgebra que contiene a $A$.
        \end{itemize}
    \end{proof}
\end{eg}

Ejercicio demostrar que la $\sigma$-álgebra de Borel está generada también por las siguientes familias:
\begin{enumerate}
    \item $\{ (a, b] : a, b \in \R, a < b \}$.
    \item $\{ [a, b) : a, b \in \R, a < b \}$.
    \item $\{ [a, b] : a, b \in \R, a < b \}$.
    \item $\{ [a, +\infty) : a \in \R \}$.
    \item $\{ (-\infty, a) : a \in \R \}$.
    \item $\{ (-\infty, a] : a \in \R \}$.
\end{enumerate}

Luego, se puede ver que $\{ a \} = \bigcap_{n \geq 1} [a, a - \frac{1}{n}) \in \mathcal{B}$.