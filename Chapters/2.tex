\section{La $\sigma$-álgebra de Borel}
\label{sec:borel}

A $\R^n$ lo pensamos dotado de la distancia euclídea.
Si $x = (x_1, \cdots, x_n)$ e $y = (y_1, \cdots, y_n)$ son dos puntos de $\R^n$, la distancia
entre ellos es
\begin{equation*}
    d(x, y) = \|x-y\| = \sqrt{ \sum_{i=1}^n (x_i - y_i)^2 }
\end{equation*}

Consideramos la topología usual de $\R^n$ notada $\tau^n$ al conjunto de todos los abiertos de $\R^n$

\begin{definition}
    Dados $a = (a_1, \cdots, a_n)$, $b = (b_1, \cdots, b_n) \in \R^n$ con $a_i < b_i (\forall i = 1, \cdots, n)$
    Definimos el intervalo abierto (a, b) como
    \begin{align*}
        (a, b) & = \prod_{i=1}^n (a_i\text{, } b_i)                                                                                                \\
               & = \{ x = (x_1\text{, } \cdots \text{, } x_n) \in \R^n : a_i < x_i < b_i \text{, } (\forall i = 1 \text{, } \cdots \text{, } n) \}
    \end{align*}
\end{definition}

\begin{definition}[$\e$-cubo]
    Dados $x= (x_1, \cdots, x_n)$ y $\e > 0$ el $\e$-cubo centrado en $x$ es el conjunto definido por
    \begin{equation*}
        C(x, \e) = \prod_{i = 1}^n (x - \frac{\e}{2}, x + \frac{\e}{2})
    \end{equation*}
\end{definition}

\begin{prop}
    Sea $V \subseteq \R^n$ abierto e $y \in C(x, \e)$ entonces
    \begin{enumerate}
        \item $(\forall x \in V) \quad (\exists \e > 0) : C(x \text{, } \e) \subseteq V$.
        \item $x \in C(y\text{, } \e)$.
        \item $C(x\text{, } \e) \subseteq C(y\text{, } 2 \cdot \e)$.
    \end{enumerate}
\end{prop}

\clearpage

\begin{definition}[$\sigma$-álgebra de Borel de $\R^n$]
    Es la $\sigma$-álgebra generada por:
    \begin{equation*}
        \mc{A} = \{ (a, b) : a\text{, } b \in \R^n : a_i < b_i \text{, } i = 1\text{, } \cdots \text{, } n \}
    \end{equation*}
    Lo notamos $\mc{B}^n$.
\end{definition}

Queremos ver que efectivamente $\tau_n \subseteq \mc{B}^n$.
Consideremos la clase $\beta_n = \{ C(q\text{, } \frac{1}{m}) : q \in \Q^n\text{, } m \in \N \}$.
$\beta_n$ es numerable pues el conjunto de índices que enumera a $\beta_n$ es
\[ \underbrace{\Q^n \times \cdots \times \Q^n}_{n \text{ veces}} \times \N \] que es numerable.

\begin{prop}
    Dado un abierto no vacío $V \subseteq \R^n$ existe una familia $\mc{A}_V \subseteq \mc{B}_n$ tal que
    $V = \bigcup_{B \in \mc{A}_V} B$.

    \begin{proof}
        Sabemos que $\Q^n$ es denso en $\R^n$. Como $V$ es abierto y no vacío entonces $V \cap \Q^n \neq \O$.
        Luego $B(x\text{, } \e) \subseteq V$ y $B(x\text{, } \e) \cap \Q^n \neq \O$. Por lo tanto $B(x\text{, } \e) \subset V \cap \Q^n$.

        Para cada $q \in V \cap \Q^n$ defino $m_q = \min\{ m \in \N : C(q, \frac{1}{m}) \} \subseteq V$.
        Llamemos $\mc{A}_V = \{ C(q\text{, } \frac{1}{m_q}) : q \in V \cap \Q^n \}$ la cual es una familia numerable.

        Veamos que $\bigcup_{q \in V \cap \Q^n} C(q\text{, } \frac{1}{m_q}) = V$.
        \begin{itemize}
            \item $\subseteq$ es trivial.
            \item $\supseteq$ Dado $x \in V$, $\exists m \in \N : C(x, \frac{1}{m}) \subseteq V$. Consideremos
                  $C(x\text{, } \frac{1}{2m}) \subseteq C(x, \frac{1}{m}) \subseteq V$ que es un abierto no vacío.

                  Resulta que $C(x\text{, } \frac{1}{2m}) \cap \Q^n \neq \O$.
                  Sea $q \in C(x\text{, } \frac{1}{2m}) \subseteq V \cap \Q^n$

                  $\Rightarrow x \in C(q\text{, } \frac{1}{2m})$, en particular $m_q \leq 2m$, pues como $x \in C(q\text{, } \frac{1}{2m})$
                  implica que $C(q\text{, } \frac{1}{2m}) \subseteq C(x\text{, } \frac{2}{2m}) \subseteq V$.

                  $\Rightarrow x \in C(q\text{, } \frac{1}{2m}) \subseteq C(q\text{, } \frac{1}{m_q}) \therefore x \in \bigcup_{q \in \mc{A}_v} C(q, \frac{1}{m_q}) = \bigcup_{B \in \mc{A}_V} B$ .
        \end{itemize}
    \end{proof}
\end{prop}

\begin{corollary}
    La $\sigma$-álgebra de Borel de $\R^n$ coincide con la $\sigma(\tau_n)$. En particular:
    \begin{itemize}
        \item Todo abierto de $\R^n$ es un conjunto Boreliano.
        \item Todo conjunto cerrado de $\R^n$ es un Boreliano por ser complemento de un abierto.
        \item Por último, todo subconjunto numerable de $\R^n$ es un Boreliano. (Dado $x \in \R^n, \{ x \} = \bigcap_{n \geq 1}C(x, \frac{1}{n})$).
    \end{itemize}
\end{corollary}

\clearpage

\begin{prop}
    Dado un espacio medible $(X\text{, } \mf{X})$ y sea $X_0 \subseteq \mf{X}$, entonces
    \begin{enumerate}
        \item $\mf{X}_0 = \{ A \subseteq X_0 : A = E \cap X_0 \text{ para algún } E \in \mf{X} \}$ es $\sigma$-álgebra de $X_0$. En particular,
              si $X_0 \in \mf{X} \Rightarrow \mf{X}_0 = \{ A \subseteq X_0 : A \in \mf{X} \}$, la demostración queda como ejercicio.
        \item Si $\mc{A}$ es una familia en partes de $X$ tal que $\mf{X} = \sigma(\mc{A})$ entonces $\mf{X}_0 = \sigma(\mc{A}_0)$ donde
              $\mc{A}_0 = \{ A_0 \subseteq X_0 : A_0 = A \cap X_0 \text{ para algún } A \in \mc{A} \}$.
              \begin{proof}
                  Veamos primero que $\mc{A}_0 \subseteq \mf{X}_0$. Si $A_0 \in \mc{A}_0 \Rightarrow \exists A \in \mc{A} : A_0 = \mc{A} \cap X_0$.
                  Como $A \in \mc{A} \subseteq \sigma(\mc{A}) = \mf{X}$ resulta que $A_0 = A \cap X_0 \in \mf{X}_0$.
                  Entonces $\mc{A}_0 \subseteq \mf{X}_0$. Por lo tanto $\sigma(\mc{A}_0) \subseteq \mf{X}_0$.

                  Ahora veamos que $\mf{X}_0 \subseteq \sigma(\mc{A}_0)$.
                  Consideramos la clase $\mc{G} = \{ E \subseteq X : E \cap X_0 \in \sigma(\mc{A}_0) \}$ y veamos que $\mf{X} \subseteq \mc{G}$.
                  Alcanza con probar que $\mc{A} \subseteq \mc{G}$. Pues si $A \in \mc{A}$, $A \cap X_0 \in \mc{A}_0 \subseteq \sigma(\mc{A}_0) \Rightarrow A \in \mc{G}$.
                  Si probamos que $G$ es una $\sigma$-álgebra, tendríamos que $\mc{A} \subseteq \mc{G}$ y $\sigma(\mc{A}) \subseteq \mc{G}$ y $\sigma(\mc{A}) = \mf{X}$.
                  Luego $\mf{X}_0 \subseteq \sigma(\mc{A}_0)$.
              \end{proof}
    \end{enumerate}
\end{prop}

\begin{eg}
    Si $\beta \in B_n$ entonces la $\sigma$-álgebra de Borel de $\beta$, $B_n(\beta) = \{ A \subseteq \beta : A \in B_n \}$ está generado por la familia de conjuntos
    de la forma $(a, b) \cap \beta$ para $a$, $b \in \R^n$ con $a_i < b_i \quad (\forall i = 1 \text{, } \cdots \text{, } n)$.
\end{eg}

\section{Recta real extendida}

\begin{definition}[Recta real extendida]
    Definimos $\overline{\R} = \R \cup \{ -\infty, +\infty \}$. Con las siguientes convenciones:
    \begin{enumerate}
        \item Dado $r \in \R$ tenemos que $-\infty < r < +\infty$.
        \item $\substack{+ \\ -} \infty + \substack{+ \\ -} \infty = \substack{+ \\ -} \infty$ y
              $\substack{+ \\ -} \infty + \substack{- \\ +} \infty$ no está definido.
        \item $\substack{+ \\ -} \infty \cdot \substack{+ \\ -} \infty = +\infty$ y
              $\substack{+ \\ -} \infty \cdot \substack{- \\ +} \infty = -\infty$
              Si $r \in \R$ entonces $r \cdot +\infty = +\infty$ si $r > 0$ y $r \cdot +\infty = -\infty$. si $r < 0$.
        \item $0 \cdot \substack{+ \\ -} \infty = 0 = \substack{+ \\ -} \infty \cdot 0$.
        \item Tampoco definimos cocientes entre infinitos o de la forma $\frac{r}{\substack{+ \\ -} \infty}$.
    \end{enumerate}
\end{definition}

\begin{note}
    El producto no va a ser continuo en la recta real extendida.
    Si $a_n = +\infty \cdot \frac{1}{n} (\forall n \in \N)$ entonces $lim_{n \to +\infty} a_n = +\infty$. Pero $+\infty \cdot lim_{n \to +\infty} \frac{1}{n} = +\infty \cdot 0 = 0$.
\end{note}

Notemos que si $A \subseteq \overline{\R} \to inf(A) \in \overline{\R}$ y $sup(A) \in \overline{\R}$.

Dada una sucesión $(x_n)_{n \in \N} \subseteq \R$, sea $\O \neq L = \{ x \in \overline{\R} : \exists x_{n_k} \to x \} \subseteq \overline{\R}$.

\begin{definition}
    $\limsup_{n \to \infty} x_n = sup(L)$ y $\liminf_{n \to \infty} x_n = inf(L)$.Ambos pertenecen a $L$.

    Además, si para cada $n \in \N$ definimos $\alpha_m = sup\{ x_n : n \geq m \}$ la sucesión $\alpha_m$ es decreciente y
    $\limsup_{n \to \infty} x_n = inf\{\alpha_m\} = inf_{m \geq n} ( sup_{n \geq m}\{x_n\} )$.

    Análogamente $\liminf_{n \to \infty} x_n = sup\{\alpha_m\} = sup_{m \geq n} ( inf_{n \geq m}\{x_n\} )$.
\end{definition}

\begin{prop}
    Propiedades de límite superior e inferior:
    \begin{itemize}
        \item $\limsup_{n \to \infty} (- x_n) = - \liminf_{n \to \infty} x_n$
        \item $\liminf_{n \to \infty} (- x_n) = - \limsup_{n \to \infty} x_n$
    \end{itemize}
\end{prop}

\begin{note}
    Si $(x_n)_{n \in \N}$ es una sucesión en $\R$ y $x \in \overline{\R}, x_n \to x \iff \limsup x_n = \liminf x_n = x$.
\end{note}

Veamos como extender $\mc{B}$ a $\overline{\R}$.

\begin{definition}[Borel extendida]
    Para cada $E \in \mc{B}$, sean $E_1 = E \cup \{ + \infty \}$, $E_2 = E \cup \{ - \infty \}$ y $E_3 = E \cup \{ + \infty, - \infty \}$.
    Consideremos $\overline{\mc{B}} = \{ E_1 \text{, } E_2 \text{, } E_3 \text{, } E : E \in \mc{B} \} = \sigma(\{ (a, +\infty] : a \in \R \})$.
    Probar que $\overline{\mc{B}}$ es $\sigma$-álgebra de $\overline{\R}$ se deja como ejercicio.
\end{definition}

