\section{Funciones medibles}

\begin{prop}
    Sea $f: \R^n \to \R$, $f$ es continua si $f^{-1}(V)$ es abierto de $ \R^n$ $(\forall V$ abierto en $\tau_1)$.
\end{prop}
En lo que sigue vamos a considerar un espacio medible fijo de la forma $(X, \mf{X})$.

Notación: Dada una función $f: X \to \R$ para cada $\alpha \in \R$ definimos: \begin{equation}
    \{ f > \alpha \} := \{ x \in X : f(x) > \alpha \} = f^{-1}((\alpha, +\infty))
\end{equation}

\begin{definition}[Función medible]
    Una función $f: X \to \R$ es $\mf{X}$-medible ($ \sigma $-medible) si $\{ f > \alpha \} \in \mf{X} \quad (\forall \alpha \in \R)$.
\end{definition}

\clearpage

\begin{lemma}
    Dada $f: X \to \R$ una función, son equivalentes:
    \begin{enumerate}
        \item $f$ es $\mf{X}$-medible.
        \item $\{ f \geq \alpha \} \in \mf{X} \quad (\forall \alpha \in \R)$.
        \item $\{ f \leq \alpha \} \in \mf{X} \quad (\forall \alpha \in \R)$.
        \item $\{ f < \alpha \} \in \mf{X} \quad (\forall \alpha \in \R)$.
    \end{enumerate}

    \begin{proof}
        Sea $\alpha, \gamma \in \R$.
        \begin{itemize}
            \item $(1) \iff (3)$: $\{ f \leq \alpha \} = \{ f > \alpha \}^c \in \mf{X}$.
            \item $(2) \iff (4)$ Análogo.
            \item $(1) \iff (2)$: Supongamos que $f$ es $\mf{X}$-medible. Dado $\alpha \in \R$ quiero ver que $\{ f \geq \alpha \} \in \mf{X}$.
                  \begin{equation}
                      x \in \{ f \geq \alpha \} \iff f(x) \geq \alpha > \alpha - \frac{1}{n} \quad (\forall n \in \N)
                  \end{equation}
                  \begin{equation}
                      x \in \{ f > \alpha - \frac{1}{n} \} \quad (\forall n \in \N) \to \{ f \geq \alpha \} = \bigcap_{n \geq 1} \{ f > \alpha - \frac{1}{n} \} \in \mf{X}
                  \end{equation}
                  Para la vuelta supongamos que vale $(2)$. Quiero ver que $\{ f > \gamma \} \in \mf{X}$. Notemos que
                  \begin{equation}
                      \{ f > \gamma \} = \bigcup_{n \geq 1} \{ f \geq \gamma + \frac{1}{n} \}
                  \end{equation}
                  \begin{equation}
                      x \in \{ f > \gamma \} \iff f(x) > \gamma \iff \exists n_x \in \N : f(x) > \gamma + \frac{1}{n_x}
                  \end{equation}
                  Luego $\bigcup_{n \geq 1} \{ f \geq \gamma + \frac{1}{n} \} \in \mf{X}$.
        \end{itemize}
    \end{proof}
\end{lemma}

\begin{eg}
    Toda función constante es medible. $f: X \to \R : f(x) = c \quad (\forall x \in X)$.
    \begin{proof}
        Dado $\alpha \in \R$. Tenemos que \begin{equation}
            \{ f > \alpha \} = \begin{cases}
                \O & \alpha \geq c \\
                X  & \alpha < c
            \end{cases}
        \end{equation}
    \end{proof}
\end{eg}

\begin{eg}
    Dado $E \subseteq X$ consideremos la función característica de $E$. Como $\chi_E: X \to \R$ dada por:
    \begin{equation}
        \chi_E(x) = \begin{cases}
            1 & x \in E    \\
            0 & x \notin E
        \end{cases}
    \end{equation}
    \begin{proof}
        Consideremos $E = [0, 1]$. Dado $\alpha \in \R$. Tenemos que \begin{equation}
            \{ \chi_E > \alpha \} = \begin{cases}
                \O & \alpha \geq 1     \\
                E  & 0 \leq \alpha < 1 \\
                X  & \alpha < 0
            \end{cases}
        \end{equation}
        Luego $\chi_E$ es medible $\iff E \in \mf{X}$.
    \end{proof}
\end{eg}

\begin{eg}
    Si $X = \R^n$, $\mf{X} = \mathcal{B} \to$ toda función continua es medible con respecto a la $\sigma$-álgebra de Borel.
\end{eg}

\begin{eg}
    $X = \R, \mf{X} = \mathcal{B}_n$ y $f: \R \to \R$ es monótona (creciente) entonces es $\mathcal{B}$-medible.
\end{eg}

Ejercicio: $f: X \to \R$ es $\mf{X}$-medible $\iff f^{-1}(B) \in \mf{X} \quad (\forall B \in \mathcal{B})$.

\clearpage

\begin{lemma}
    Sean $f, g: X \to \R$ funciones medibles, $c \in \R$ entonces $c \cdot f$, $f²$, $f + g$, $|f|$, $f \cdot g$, son $\mf{X}$-medibles.
    $f² = f(x) \cdot f(x)$.
    \begin{proof}
        Veamos que $f²$ es $\mf{X}$-medible. Dado $\alpha \in \R$ quiero ver que \begin{equation}
            \{ f² > \alpha \} \in \mf{X}
        \end{equation}
        Si $\alpha < 0 \to \{ f² > \alpha \} = X$.

        Si $\alpha \geq 0 \to$
        \begin{equation}
            \{ f^2 > \alpha \} = \{ x \in X : f(x) \cdot f(x) > \sqrt{\alpha} \cdot \sqrt{\alpha} \}
        \end{equation}
        \begin{equation}
            \{ f > \sqrt{\alpha} \} \cup \{ f > \sqrt{\alpha} \} \in \mf{X}
        \end{equation} $\therefore f²$ es $\mf{X}$-medible.

        Veamos ahora que $f + g$ es $\mf{X}$-medible. Dado $\alpha \in \R$ quiero ver que \begin{equation}
            \{ f + g > \alpha \} \in \mf{X}
        \end{equation}
        Para $x \in X$ tenemos que: \begin{equation}
            (f+g)(x) > \alpha \iff f(x) + g(x) > \alpha \iff f(x) > r \land g(x) > \alpha - r \text{ para algún } r \in \Q
        \end{equation}
        Entonces $\{ f + g > \alpha \} = \bigcup_{r \in \Q} (\{ f > r \} \cap \{ g > \alpha - r \}) \in \mf{X}$ por ser unión numerable $\therefore f+g$ es $\mf{X}$-medible.

        Por último veamos que $f \cdot g$ es $\mf{X}$-medible. Dado $\alpha \in \R$ quiero ver que \begin{equation}
            \{ f \cdot g > \alpha \} \in \mf{X}
        \end{equation}
        Sabemos que:
        \begin{equation}
            (f+g)² \text{ es } \mf{X} \text{-medible} \to f² + 2 \cdot f \cdot g + g² \text{ es } \mf{X} \text{-medible}
        \end{equation}
        \begin{equation}
            f \cdot g = \frac{1}{2} \left( (f+g)² - f² - g² \right) \text{ es } \mf{X} \text{-medible}
        \end{equation}
    \end{proof}
\end{lemma}

\section{Funciones medibles en la recta extendida}

\begin{definition}
    Dada $f: X \to \overline{R}$ diremos que $f$ es $\mf{X}$-medible si
    \begin{equation}
        f^{-1}((\alpha, +\infty]) = f^{-1}((\alpha, +\infty)) \cup f^{-1}(\{ +\infty \}) = \{ f > \alpha \} \in \mf{X} \quad (\forall \alpha \in \R)
    \end{equation}
\end{definition}

A la clase de las funciones (a valores en la recta extendida) $\mf{X}$-medibles la denotaremos por $M(X, \mf{X})$.

\begin{note}
    Si $f: X \to \R \to f \in M(X, \mf{X})$.
\end{note}

\begin{note}
    Si \begin{equation}f \in M(X, \mf{X}) \to \{ f = +\infty \} = f^{-1}(\{ +\infty \}) = \bigcap_{n \geq 1} \{ f > n \} \in \mf{X}\end{equation}
    Además, \begin{equation}\{ f = -\infty \} = f^{-1}(\{ -\infty \}) = \bigcap_{n \geq 1} \{ f < -n \} \in \mf{X}\end{equation}
\end{note}

\begin{lemma}
    Dada una función $f: X \to \overline{\R}$ consideremos $A_f = \{ f = +\infty \}$, $B_f = \{ f = -\infty \}$ y
    \begin{equation} \hat{f} = \begin{cases}
            f & x \in X \setminus (A_f \cup B) \\
            0 & x \in A_f                      \\
            0 & x \in B_f
        \end{cases}
    \end{equation}
    $\to f \in M(X, \mf{X}) \iff A_f, B_f \in \mf{X}$ y $\hat{f}$ es $\mf{X}$-medible.
    \begin{proof}
        Supongamos primero que $f \in M(X, \mf{X})$. Dado $\alpha \in \R$, ya vimos que $A_f, B_f \in \mf{X}$. Veamos que $\hat{f}$ es $\mf{X}$-medible.

        Quiero ver que $\{ \hat{f} > \alpha \} \in \mf{X}$. Si $\alpha \geq 0$ entonces
        \begin{equation}
            \{ \hat{f} > \alpha \} = \{ f > \alpha \} - A_f = \{ f > \alpha \} \cap A_f^C \in \mf{X}
        \end{equation}
        Si $\alpha < 0$ entonces \begin{equation}
            \{ \hat{f} < \alpha \} = \{ f > \alpha \} \cup \{ \hat{f} = 0 \} = \{ f > \alpha \} \cup (A_f \cup B_f) = \{ f > \alpha\} \cup B_f \in \mf{X}
        \end{equation}
        Luego $\hat{f} \in \mf{X}$. Supongamos ahora que $A_f, B_f \in \mf{X}$ y $\hat{f}$ es $\mf{X}$-medible. Dado $\alpha \in \R$ tenemos que \begin{equation}
            \{ f > \alpha \} = \{ \hat{f} > \alpha \} \cup A_f \in \mf{X}
        \end{equation}
        Si $\alpha < 0$ entonces \begin{equation}
            \{ f > \alpha \} = \{ \hat{f} < \alpha \} \setminus B_f = \{ \hat{f} < \alpha \} \cap B_f^c \in \mf{X}
        \end{equation}
    \end{proof}
\end{lemma}

\begin{corollary}
    Si $f, g \in M(X, \mf{X})$ y $c \in \R$. Las funciones $c \cdot f$, $f²$, $|f|$, $f \cdot g \in M(X, \mf{X})$.
\end{corollary}

\clearpage

\begin{note}
    Dados $f, g \in M(X, \mf{X})$ consideremos los conjuntos \begin{itemize}
        \item $E_1 = \{ f = + \infty \} \cap \{ g = - \infty \} \in \mf{X}$.
        \item $E_2 = \{ f = - \infty \} \cap \{ g = + \infty \} \in \mf{X}$.
    \end{itemize}
    Notemos que no está definida la suma $f + g$ en $E_1 \cup E_2$.
    Definimos \begin{equation}
        f+g = \begin{cases}
            f+g & x \in X \setminus (E_1 \cup E_2) \\
            0   & x \in E_1 \cup E_2
        \end{cases}
    \end{equation}
    La demostración de que $f+g \in M(X, \mf{X})$ se deja como ejercicio.
\end{note}

\begin{lemma}
    Dada una sucesión de funciones $(f_n)_{n \geq 1}$ en $M(X, \mf{X})$ sean $f, f^*, F, F^*$ definidas por:
    \begin{equation}
        f(x) = inf_{n \geq 1} f_n(x) \quad f^*(x) = \liminf_{n \to \infty} f_n(x)
    \end{equation}
    \begin{equation}
        F(x) = sup_{n \geq 1} f_n(x) \quad F^*(x) = \limsup_{n \to \infty} f_n(x)
    \end{equation}
    Entonces $f, f^*, F, F^* \in M(X, \mf{X})$.
    \begin{proof}
        Dado $\alpha \in \R$ tenemos que \begin{equation}
            \{ F > \alpha \} = \bigcup_{n \geq 1} \{ f_n > \alpha \} \in \mf{X}
        \end{equation}
        \begin{equation}
            \{ f > \alpha \} = \bigcap_{n \geq 1} \{ f_n > \alpha \} \in \mf{X}
        \end{equation}
        Veamos $F^* \in M(X, \mf{X})$. Para cada $n \in \N$ defino $h_n = sup_{m \geq n} f_m \in \mf{X}$. Por ser subsucesión de funciones medibles.
        Luego \begin{equation}
            F^* = inf_{n \geq 1} (sup_{m \geq n} f_m) \in \mf{X}
        \end{equation}
        Análogamente para $f^*$.
    \end{proof}
\end{lemma}

\begin{corollary}
    Dada $(f_n)_{n \geq 1} : f_n \in M(X, \mf{X}) \quad (\forall n \in \N)$. Supongamos que la sucesión converge puntualmente a $f$ entonces $f \in M(X, \mf{X})$.
    \begin{proof}
        Notemos que $f = \liminf f_n = \limsup f_n$ y aplicamos el lema anterior.
    \end{proof}
\end{corollary}